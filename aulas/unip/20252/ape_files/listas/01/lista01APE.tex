\documentclass[12pt]{exam}

\usepackage[brazil]{babel}   
\usepackage[TS1,T1]{fontenc}
\usepackage[utf8]{luainputenc}
\usepackage[utf8]{inputenc}  
%\usepackage[T1]{fontenc}




\usepackage[margin=1in]{geometry}
\usepackage{amsmath,amssymb}
\usepackage{multicol}

\def\code#1{\texttt{#1}}

\newcommand{\class}{APE}
\newcommand{\term}{Algoritmos Para Engenheiros}
\newcommand{\examnum}{Lista 1}
\newcommand{\examdate}{}
\newcommand{\timelimit}{}

\pointpoints{ponto}{pontos}

\pagestyle{head}
\firstpageheader{}{}{}
\runningheader{\class}{\examnum\ - Página \thepage\ de \numpages}{\examdate}
\runningheadrule


\begin{document}

\noindent
\begin{tabular*}{\textwidth}{l @{\extracolsep{\fill}} r @{\extracolsep{6pt}} l}
\textbf{\class}   &  & \\
\textbf{\term}    &  & \\
\textbf{\examnum} &  & \\
                  & Professor: & Vinicius Pereira
\end{tabular*}\\
\rule[2ex]{\textwidth}{2pt}



\begin{questions}


\question Faça um programa que peça para o usuário digitar 
um número. O seu programa deve calcular o valor absoluto (o módulo)
deste número e exibir para o usuário.


\makeemptybox{\stretch{1}}

\question Faça um programa que peça para o usuário digitar 
um número. 
O seu programa deve dizer se o número digitado 
é par ou impar.


\makeemptybox{\stretch{1}}

\break



\question Faça um programa que peça para o usuário digitar 
dois números. 
O seu programa deve calcular e exibir a soma dos dois números.
Assim como a subtração, divisão e multiplicação.


\makeemptybox{\stretch{1}}

\question Faça um programa que peça para o usuário digitar 
três números. 
O seu programa deve calcular a média dos três números
e mostrar o resultado.


\makeemptybox{\stretch{1}}

\break


\end{questions}


\end{document}
