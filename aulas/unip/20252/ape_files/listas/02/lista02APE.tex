\documentclass[12pt]{exam}

\usepackage[brazil]{babel}   
\usepackage[TS1,T1]{fontenc}
\usepackage[utf8]{luainputenc}
\usepackage[utf8]{inputenc}  
%\usepackage[T1]{fontenc}




\usepackage[margin=1in]{geometry}
\usepackage{amsmath,amssymb}
\usepackage{multicol}


\def\code#1{\texttt{#1}}

\newcommand{\class}{APE}
\newcommand{\term}{Algoritmos Para Engenheiros}
\newcommand{\examnum}{Lista 2}
\newcommand{\examdate}{}
\newcommand{\timelimit}{}

\pointpoints{ponto}{pontos}

\pagestyle{head}
\firstpageheader{}{}{}
\runningheader{\class}{\examnum\ - Página \thepage\ de \numpages}{\examdate}
\runningheadrule


\begin{document}

\noindent
\begin{tabular*}{\textwidth}{l @{\extracolsep{\fill}} r @{\extracolsep{6pt}} l}
\textbf{\class}   &  & \\
\textbf{\term}    &  & \\
\textbf{\examnum} &  & \\
                  & Professor: & Vinicius Pereira
\end{tabular*}\\
\rule[2ex]{\textwidth}{2pt}



\begin{questions}


\question Faça um programa que receba um número $n$ do usuário.
Usando o comando de repetição \code{while}. 
Faça um programa que imprima todos os números de $0$ a $n$.


\makeemptybox{\stretch{1}}

\question Faça um programa que receba um número $n$ do usuário.
Usando o comando de repetição \code{while}. 
Faça um programa que imprima todos os números de $n$ a $0$.


\makeemptybox{\stretch{1}}

\break






\question Faça um programa que receba um número $n$ do usuário.
Usando o comando de repetição \code{while}. 
Faça um programa que imprima todos os números pares 
maiores que 0 e menores que $n$.


\makeemptybox{\stretch{1}}

\question Faça um programa que receba um número $n$ do usuário.
Usando o comando de repetição \code{while}. 
Faça um programa que imprima os $n$ primeiros 
números pares.


\makeemptybox{\stretch{1}}

\break






\question Faça um programa que receba um número $n$ do usuário.
Usando o comando de repetição \code{for} e a função geradora \code{range}. 
Faça um programa que imprima todos os números pares 
maiores que 0 e menores que $n$.


\makeemptybox{\stretch{1}}

\question Faça um programa que receba um número $n$ do usuário.
Usando o comando de repetição \code{for} e a função geradora \code{range}. 
Faça um programa que imprima os $n$ primeiros 
números pares.


\makeemptybox{\stretch{1}}

\break






\end{questions}


\end{document}
