\documentclass[12pt]{exam}

\usepackage[brazil]{babel}   
\usepackage[TS1,T1]{fontenc}
\usepackage[utf8]{luainputenc}
\usepackage[utf8]{inputenc}  
%\usepackage[T1]{fontenc}



\usepackage{tikz}
\usetikzlibrary{arrows,shapes.gates.logic.US,shapes.gates.logic.IEC,calc}


\usepackage[margin=1in]{geometry}
\usepackage{amsmath,amssymb}
\usepackage{multicol}

\usepackage{karnaugh-map}

\def\code#1{\texttt{#1}}

\newcommand{\class}{CLD}
\newcommand{\term}{Circuitos Lógicos Digitais}
\newcommand{\examnum}{Lista 2}
\newcommand{\examdate}{}
\newcommand{\timelimit}{}

\pointpoints{ponto}{pontos}

\pagestyle{head}
\firstpageheader{}{}{}
\runningheader{\class}{\examnum\ - Página \thepage\ de \numpages}{\examdate}
\runningheadrule


\begin{document}

\noindent
\begin{tabular*}{\textwidth}{l @{\extracolsep{\fill}} r @{\extracolsep{6pt}} l}
\textbf{\class}   &  & \\
\textbf{\term}    &  & \\
\textbf{\examnum} &  & \\
                  & Professor: & Vinicius Pereira
\end{tabular*}\\
\rule[2ex]{\textwidth}{2pt}



\begin{questions}


\question 

\begin{enumerate}

\item

Considere a seguinte tabela verdade e 
monte uma expressão booleana que satisfaça a 
tabela.

\begin{displaymath}
\begin{array}{|c c|c|}
a & b & resultado \\
\hline 
0 & 0 & 0\\
0 & 1 & 0\\
1 & 0 & 1\\
1 & 1 & 1\\
\end{array}
\end{displaymath}


\makeemptybox{\stretch{1}}

\item

Ainda considerando a mesma tabela verdade, 
monte o mapa de Karnaugh e encontre uma equação 
simplificada da tabela verdade.


    \begin{karnaugh-map}[2][2][1][$a$][$b$]
    \end{karnaugh-map}


\makeemptybox{\stretch{1}}

\item

Considerando a equação simplificada do item anterior,
monte um circuito baseado nesta equação.

\makeemptybox{\stretch{1}}

\end{enumerate}



\break










\question 

\begin{enumerate}

\item

Considere a seguinte tabela verdade e 
monte uma expressão booleana que satisfaça a 
tabela.

\begin{displaymath}
\begin{array}{|c c|c|}
a & b & resultado \\
\hline 
0 & 0 & 1\\
0 & 1 & 1\\
1 & 0 & 0\\
1 & 1 & 1\\
\end{array}
\end{displaymath}


\makeemptybox{\stretch{1}}

\item

Ainda considerando a mesma tabela verdade, 
monte o mapa de Karnaugh e encontre uma equação 
simplificada da tabela verdade.


    \begin{karnaugh-map}[2][2][1][$a$][$b$]
    \end{karnaugh-map}


\makeemptybox{\stretch{1}}

\item

Considerando a equação simplificada do item anterior,
monte um circuito baseado nesta equação.

\makeemptybox{\stretch{1}}

\end{enumerate}


\break






\question 

\begin{enumerate}

\item

Considere a seguinte tabela verdade e 
monte uma expressão booleana que satisfaça a 
tabela.

\begin{displaymath}
\begin{array}{|c c c|c|}
a & b & c & resultado \\
\hline 
0 & 0 & 0 & 0\\
0 & 0 & 1 & 1\\
0 & 1 & 0 & 0\\
0 & 1 & 1 & 1\\
1 & 0 & 0 & 1\\
1 & 0 & 1 & 1\\
1 & 1 & 0 & 1\\
1 & 1 & 1 & 1\\
\end{array}
\end{displaymath}


\makeemptybox{\stretch{1}}

\item

Ainda considerando a mesma tabela verdade, 
monte o mapa de Karnaugh e encontre uma equação 
simplificada da tabela verdade.


    \begin{karnaugh-map}[4][2][1][$a b$][$c$]
    \end{karnaugh-map}


\makeemptybox{\stretch{1}}

\item

Considerando a equação simplificada do item anterior,
monte um circuito baseado nesta equação.

\makeemptybox{\stretch{1}}

\end{enumerate}


\break










\question 

\begin{enumerate}

\item

Considere a seguinte tabela verdade e 
monte uma expressão booleana que satisfaça a 
tabela.

\begin{displaymath}
\begin{array}{|c c c|c|}
a & b & c & resultado \\
\hline 
0 & 0 & 0 & 0\\
0 & 0 & 1 & 1\\
0 & 1 & 0 & 1\\
0 & 1 & 1 & 1\\
1 & 0 & 0 & 0\\
1 & 0 & 1 & 1\\
1 & 1 & 0 & 0\\
1 & 1 & 1 & 1\\
\end{array}
\end{displaymath}


\makeemptybox{\stretch{1}}

\item

Ainda considerando a mesma tabela verdade, 
monte o mapa de Karnaugh e encontre uma equação 
simplificada da tabela verdade.


    \begin{karnaugh-map}[4][2][1][$a b$][$c$]
    \end{karnaugh-map}


\makeemptybox{\stretch{1}}

\item

Considerando a equação simplificada do item anterior,
monte um circuito baseado nesta equação.

\makeemptybox{\stretch{1}}

\end{enumerate}


\break



\end{questions}


\end{document}
