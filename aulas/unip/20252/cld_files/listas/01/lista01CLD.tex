\documentclass[12pt]{exam}

\usepackage[brazil]{babel}   
\usepackage[TS1,T1]{fontenc}
\usepackage[utf8]{luainputenc}
\usepackage[utf8]{inputenc}  
%\usepackage[T1]{fontenc}



\usepackage{tikz}
\usetikzlibrary{arrows,shapes.gates.logic.US,shapes.gates.logic.IEC,calc}


\usepackage[margin=1in]{geometry}
\usepackage{amsmath,amssymb}
\usepackage{multicol}

\def\code#1{\texttt{#1}}

\newcommand{\class}{CLD}
\newcommand{\term}{Circuitos Lógicos Digitais}
\newcommand{\examnum}{Lista 1}
\newcommand{\examdate}{}
\newcommand{\timelimit}{}

\pointpoints{ponto}{pontos}

\pagestyle{head}
\firstpageheader{}{}{}
\runningheader{\class}{\examnum\ - Página \thepage\ de \numpages}{\examdate}
\runningheadrule


\begin{document}

\noindent
\begin{tabular*}{\textwidth}{l @{\extracolsep{\fill}} r @{\extracolsep{6pt}} l}
\textbf{\class}   &  & \\
\textbf{\term}    &  & \\
\textbf{\examnum} &  & \\
                  & Professor: & Vinicius Pereira
\end{tabular*}\\
\rule[2ex]{\textwidth}{2pt}



\begin{questions}


\question Faça um circuito para cada expressão booleana.

\begin{enumerate}

\item $\sim (a+b)$

\item $(\sim a \cdot b) + (b \cdot \sim c)$

\item $\sim a \cdot \sim b$

\item $(a+b)\cdot c$

\item $a + (b\cdot c)$

\end{enumerate}

\break




\tikzstyle{branch}=[fill,shape=circle,minimum size=3pt,inner sep=0pt]


\question Faça uma expressão booleana equivalente a cada um dos circuitos abaixo.

\begin{enumerate}

\item \textbf{Resposta:} $(a+b)\cdot c$

\begin{tikzpicture}[label distance=2mm]

    \node (c) at (0,0) {$c$};
    \node (b) at (1,0) {$b$};
    \node (a) at (2,0) {$a$};

    \node[or gate US, draw, logic gate inputs=nn] at ($(a)+(1,-1)$) (Or) {};
    \node[and gate US, draw, logic gate inputs=nn, anchor=input 1] at ($(Or.output)+(1,-1)$) (And) {};

    \draw (a) |- (Or.input 1);
    \draw (b) |- (Or.input 2);
    \draw (Or.output) |- (And.input 1);
    \draw (c) |- (And.input 2);
    \draw (And.output) -- ([xshift=0.5cm]And.output) node[above] {$~$};

\end{tikzpicture}



\item \textbf{Resposta:} $a + (b\cdot c)$

\begin{tikzpicture}[label distance=2mm]

    \node (a) at (0,0) {$a$};
    \node (b) at (1,0) {$b$};
    \node (c) at (2,0) {$c$};

    \node[and gate US, draw, logic gate inputs=nn] at ($(c)+(1,-1)$) (And) {};
    \node[or gate US, draw, logic gate inputs=nn, anchor=input 1] at ($(And.output)+(1,-1)$) (Or) {};

    \draw (c) |- (And.input 1);
    \draw (b) |- (And.input 2);
    \draw (And.output) |- (Or.input 1);
    \draw (a) |- (Or.input 2);
    \draw (Or.output) -- ([xshift=0.5cm]Or.output) node[above] {$~$};

\end{tikzpicture}






\item \textbf{Resposta:}  $\sim (a+b)$

\begin{tikzpicture}[label distance=2mm]

    \node (a) at (0,0) {$a$};
    \node (b) at (1,0) {$b$};

    \node[or gate US, draw, logic gate inputs=nn] at ($(b)+(1,-1)$) (Or) {};
    \node[not gate US, draw] at ($(Or.output)+(1,0)$) (Not) {};

    \draw (a) |- (Or.input 2);
    \draw (b) |- (Or.input 1);
    \draw (Or.output) |- (Not.input);
    \draw (Not.output) -- ([xshift=0.5cm]Not.output) node[above] {$~$};

\end{tikzpicture}



\item \textbf{Resposta:} $\sim a \cdot \sim b$

\begin{tikzpicture}[label distance=2mm]

    \node (a) at (0,0) {$a$};
    \node (b) at (1,0) {$b$};

    \node[not gate US, draw, rotate=-90] at ($(a)+(0,-1)$) (NotA) {};
    \draw (a) |- (NotA.input);
    
    \node[not gate US, draw, rotate=-90] at ($(b)+(0,-1)$) (NotB) {};
    \draw (b) |- (NotB.input);
    
    \node[and gate US, draw, logic gate inputs=nn] at ($(b)+(1,-2)$) (And) {};

    \draw (NotA) |- (And.input 2);
    \draw (NotB) |- (And.input 1);
    \draw (And.output) -- ([xshift=0.5cm]And.output) node[above] {$~$};

\end{tikzpicture}





\item \textbf{Resposta:} $(\sim a \cdot b) + (b \cdot \sim c)$

\begin{tikzpicture}[label distance=2mm]

    \node (a) at (0,0) {$a$};
    \node (b) at (1,0) {$b$};
    \node (c) at (2,0) {$c$};

    \node[not gate US, draw, rotate=-90] at ($(a)+(0,-1)$) (NotA) {};
    \draw (a) |- (NotA.input);
    
    \node[not gate US, draw, rotate=-90] at ($(c)+(0,-1)$) (NotC) {};
    \draw (c) |- (NotC.input);
    
    \node[and gate US, draw, logic gate inputs=nn] at ($(c)+(1,-2)$) (And1) {};
    \node[and gate US, draw, logic gate inputs=nn] at ($(c)+(1,-3)$) (And2) {};
    \node[or gate US, draw, logic gate inputs=nn, anchor=input 1] at ($(And1.output)+(1,0)$) (Or) {};

    \draw (NotC) |- (And1.input 1);
    \draw (b |- And1.input 2) node[branch] {} -- (And1.input 2);
    \draw (b)    |- (And2.input 1);
    \draw (NotA) |- (And2.input 2);
    \draw (And1.output) |- (Or.input 1);
    \draw (And2.output) -- ([xshift=0.5cm]And2.output) |- (Or.input 2);

    \draw (Or.output) -- ([xshift=0.5cm]Or.output) node[above] {$~$};

\end{tikzpicture}


\end{enumerate}


\break


\question

Considere o seguinte número binário $(1110011110110)_2$ forneça o mesmo número nas bases 4 e 16.

\makeemptybox{\stretch{1}}

\question

Considere o seguinte número hexadecimal $(AF12)_{16}$ forneça o mesmo número nas bases 4 e 2.

\makeemptybox{\stretch{1}}

\break


\question[1]

Prove seguinte equivalência $a\oplus b \equiv (a+b)\cdot \sim(a\cdot b)$. Dica: use a tabela verdade.

\makeemptybox{\stretch{1}}

\break


\end{questions}


\end{document}
