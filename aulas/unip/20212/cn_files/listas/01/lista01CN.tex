\documentclass[12pt]{article}

\usepackage[utf8]{inputenc}  
\usepackage[portuguese]{babel}   
\usepackage{enumerate}
\usepackage{listings}             % Include the listings-package
\usepackage{color}
\usepackage{graphicx}
\usepackage{amsmath}

\usepackage{enumitem}

\newlist{passo}{enumerate}{1}
\setlist[passo]{label=Passo \arabic*:}

\title{Lista 01 de Cálculo Numérico}
\date{2º Período de 2021}
\author{Turma do 3º ano}



\begin{document}


\maketitle


\vspace{3em}

\begin{enumerate}


\item Faça uma busca binária da raíz da equação $f(x) = -x + 8$ entre os valores $0$ e $10$. 
Considere uma tolerância de erro de $0.6$ para $f(x)$, ou seja, a busca deve continuar enquanto $|f(x)|>0.6$

Resposta:

\begin{passo}
\item $x_a = 0; f(x_a) = 8$, $x_b = 10; f(x_b) = -2$, $x = 5; f(x) = 3$
\item $x_a = 5; f(x_a) = 3$, $x_b = 10; f(x_b) = -2$, $x = 7.5; f(x) = 0.5$

\end{passo}

\item Faça uma busca binária da raíz da equação $f(x) = x^2 - 7x - 18$ entre os valores $0$ e $10$. 
Considere uma tolerância de erro de $0.6$ para $f(x)$, ou seja, a busca deve continuar enquanto $|f(x)|>0.6$

Resposta

\begin{tabular}{ |r|c|c|c|c|c|c| } 
 \hline
 Passo & $x_a$   & $f(x_a)$  & $x_b$      & $f(x_b)$  & $x$       & $f(x)$      \\ \hline
 1     &   0     &   -18     &   10       & 12        & 5         & -28         \\ \hline
 2     &   5     &   -28     &   10       & 12        & 7.5       & -14.25      \\ \hline
 3     &   7.5   &   -14.25  &   10       & 12        & 8.75      & -2.6875     \\ \hline
 4     &   8.75  &   -2.6875 &   10       & 12        & 8.75      &  4.265625   \\ \hline
 5     &   8.75  &   -2.6875 &   9.265625 &  4.265625 & 9.0078125 &  0.0859985  \\ \hline
\end{tabular}

\break


\item Triangularize e resolva sistema linear

\begin{align*} 
  3 x_1 + 2 x_2 + 4 x_3  &= 1  \\ 
    x_1 +   x_2 + 2 x_3  &= 2  \\ 
  4 x_1 + 3 x_2 - 2 x_3  &= 3
\end{align*}

Resposta: 

Eliminando a coluna 1 abaixo da linha 1

Subtraindo 
$(3 x_1 + 2 x_2 + 4 x_3 = 1)(\div 3)(\times a_{i1})$  de todas as linhas abaixo da 1

\begin{align*} 
  3 x_1 +           2 x_2 +            4 x_3  &=          1  \\ 
  0 x_1 + \frac{1}{3} x_2 + \frac{ 2}{3} x_3  &= \frac{5}{3}  \\ 
  0 x_1 + \frac{1}{3} x_2 - \frac{22}{3} x_3  &= \frac{5}{3}  \\ 
\end{align*}

Eliminando a coluna 2 abaixo da linha 2

Subtraindo 
$(0 x_1 + \frac{1}{3} x_2 + \frac{ 2}{3} x_3 = \frac{5}{3})(\div \frac{1}{3})(\times a_{i2})$  de todas as linhas abaixo da 2

\begin{align*} 
  3 x_1 +           2 x_2 +            4 x_3  &=          1  \\ 
  0 x_1 + \frac{1}{3} x_2 + \frac{ 2}{3} x_3  &= \frac{5}{3} \\ 
  0 x_1 +           0 x_2 -            8 x_3  &=          0  \\ 
\end{align*}

Vemos que 
\[-8x_3 = 0 \Rightarrow x_3 = 0\]
\[\frac{1}{3}x_2 + 0 = \frac{5}{3} \Rightarrow x_2 = 5\]
\[3x_1 + 10 + 0 = 1 \Rightarrow x_1 = - 3\]


\item Triangularize e resolva sistema linear

\begin{align*} 
  6 \alpha_0 -  2 \alpha_1 + 10 \alpha_2  &=  -50    \\ 
 -2 \alpha_0 + 10 \alpha_1 - 14 \alpha_2  &=   42   \\ 
 10 \alpha_0 - 14 \alpha_1 + 34 \alpha_2  &= -102 
\end{align*}

\break

\item Considere os pontos $(-2, -10)$, $(-2, -14)$, $(0, -9)$, $( 0, -11)$, $(1, 0)$ e $(1, -6)$. 
Faça uma regressão pelo método dos quadrados mínimos para encontrar o polinômio de grau 2 que melhor se aproxima dos pontos dados.
Encontre os valores de $\alpha_0$, $\alpha_1$ e $\alpha_2$ que melhor aproxima a função 
$f(x) = \alpha_2x^2 + \alpha_1x + \alpha_0$ dos pontos dados.

Resposta: 

Temos os pontos 

\begin{tabular}{ |c|c| } 
 \hline
  $x$ & $f(x)$ \\ \hline
 $-2$ & $-10$  \\ \hline
 $-2$ & $-14$  \\ \hline 
 $ 0$ & $-9$   \\ \hline 
 $ 0$ & $-11$  \\ \hline 
 $ 1$ & $ 0$   \\ \hline 
 $ 1$ & $-6$   \\ \hline  
\end{tabular}


Considerando $g_0(x) = 1$, $g_1(x) = x$ e $g_2(x) = x^2$ temos


\begin{tabular}{ |c|c|c|c|c|c| } 
 \hline
 indice &  $x$ & $f(x)$ & $g_0(x)$ & $g_1(x)$ & $g_2(x)$ \\ \hline
 $0$ & $-2$ & $-10$ & $1.0$ & $-2$ & $4$ \\ \hline
 $1$ & $-2$ & $-14$ & $1.0$ & $-2$ & $4$ \\ \hline
 $2$ &  $0$ &  $-9$ & $1.0$ &  $0$ & $0$ \\ \hline
 $3$ &  $0$ & $-11$ & $1.0$ &  $0$ & $0$ \\ \hline
 $4$ &  $1$ &   $0$ & $1.0$ &  $1$ & $1$ \\ \hline
 $5$ &  $1$ &  $-6$ & $1.0$ &  $1$ & $1$ \\ \hline
\end{tabular}

Considerando 
\[a_{ij} = \sum_{k=0}^5 g_i(x_k)*g_j(x_k)\]
e 
\[b_i = \sum_{k=0}^5 f(x_k)*g_i(x_k)\]
para montar o sistema linear 


\begin{align*} 
a_{00}\alpha_0 + a_{01}\alpha_1 + a_{02}\alpha_2  &=  b_0 \\ 
a_{10}\alpha_0 + a_{11}\alpha_1 + a_{12}\alpha_2  &=  b_1 \\ 
a_{20}\alpha_0 + a_{21}\alpha_1 + a_{22}\alpha_2  &=  b_2 
\end{align*}

Ficamos com o sistema linear 

\begin{align*} 
  6 \alpha_0 -  2 \alpha_1 + 10 \alpha_2  &=  -50    \\ 
 -2 \alpha_0 + 10 \alpha_1 - 14 \alpha_2  &=   42   \\ 
 10 \alpha_0 - 14 \alpha_1 + 34 \alpha_2  &= -102 
\end{align*}

Resultando em 

$\alpha_0 = -10$, 
$\alpha_1 = 5$ e 
$\alpha_2 = 2$.

A função que melhor aproxima a curva é $2x^2 + 5x - 10$



\end{enumerate}

\end{document}









