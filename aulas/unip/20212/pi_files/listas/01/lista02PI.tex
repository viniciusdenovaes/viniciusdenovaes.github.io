\documentclass[12pt]{article}

\usepackage[utf8]{inputenc}  
\usepackage[portuguese]{babel}   
\usepackage{enumerate}
\usepackage{listings}             % Include the listings-package
\usepackage{color}
\usepackage{graphicx}
\usepackage{amsmath}


\title{Lista 02 de Processamento de Imagens}
\date{2º Período de 2018}
\author{Turma do 3º ano}



\begin{document}


\maketitle


\vspace{3em}


A descrição de cada filtro usado aqui, e cada máscara, pode ser encontrada na página do trabalho.

\begin{enumerate}

\item As seguintes imagens têm 36 pixels de largura e de altura. 
Cada linha das figuras têm mais de 3 pixels de largura. 
Passe cada máscara do filtro de Sobel em cada imagem e em cada matriz e descreva o resultado.
Descreva o resultado de passar o filtro de Sobel nas imagens e matrizes.

\begin{figure}[ht]
    \centering
    \includegraphics[width=0.20\textwidth]{quadrado36}
    \caption{Quadrado 36x36}
    \label{fig:quadrado36}
\end{figure}

\begin{figure}[ht]
    \centering
    \includegraphics[width=0.20\textwidth]{triangulo36}
    \caption{Quadrado 36x36}
    \label{fig:quadrado36}
\end{figure}

\begin{figure}[ht]
    \centering
    \includegraphics[width=0.20\textwidth]{horizontal_baixo36}
    \caption{Quadrado 36x36}
    \label{fig:quadrado36}
\end{figure}

\begin{figure}[ht]
    \centering
    \includegraphics[width=0.20\textwidth]{horizontal_cima36}
    \caption{Quadrado 36x36}
    \label{fig:quadrado36}
\end{figure}

\begin{figure}[ht]
    \centering
    \includegraphics[width=0.20\textwidth]{vertical_direita36}
    \caption{Quadrado 36x36}
    \label{fig:quadrado36}
\end{figure}

\begin{figure}[ht]
    \centering
    \includegraphics[width=0.20\textwidth]{vertical_esquerda36}
    \caption{Quadrado 36x36}
    \label{fig:quadrado36}
\end{figure}

\[
\begin{bmatrix}
  0 &  0 &  0 &  0 &  0 & 10 \\
  0 &  0 &  0 &  0 & 10 & 10 \\
  0 &  0 &  0 & 10 & 10 & 10 \\
  0 &  0 & 10 & 10 & 10 & 10 \\
  0 & 10 & 10 & 10 & 10 & 10 \\
 10 & 10 & 10 & 10 & 10 & 10 
\end{bmatrix}
\]

\[
\begin{bmatrix}
  0 &  0 &  0 & 10 & 10 & 10 \\
  0 &  0 &  0 & 10 & 10 & 10 \\
  0 &  0 &  0 & 10 & 10 & 10 \\
  0 &  0 &  0 & 10 & 10 & 10 \\
  0 &  0 &  0 & 10 & 10 & 10 \\
  0 &  0 &  0 & 10 & 10 & 10 
\end{bmatrix}
\]

\[
\begin{bmatrix}
  0 &  0 &  0 &  0 &  0 &  0 \\
  0 &  0 &  0 &  0 &  0 &  0 \\
  0 &  0 &  0 &  0 &  0 &  0 \\
 10 & 10 & 10 & 10 & 10 & 10 \\
 10 & 10 & 10 & 10 & 10 & 10 \\
 10 & 10 & 10 & 10 & 10 & 10 
\end{bmatrix}
\]

\break

\item Passe o filtro de contraste da matriz abaixo.


\[
\begin{bmatrix}
  0 &  0 &  2 &  6 &  2 &  0 \\
  0 &  0 &  2 &  6 &  2 &  0 \\
  0 &  0 &  2 &  6 &  2 &  0 \\
  0 &  0 &  2 &  6 &  2 &  0 \\
  0 &  0 &  2 &  6 &  2 &  0 \\
  0 &  0 &  2 &  6 &  2 &  0 
\end{bmatrix}
\]

\item Passe o filtro blur em todas as imagens e matrizes da lista.


\end{enumerate}

\end{document}









