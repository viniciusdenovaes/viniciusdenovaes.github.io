\documentclass[12pt]{exam}

\usepackage[brazil]{babel}   
\usepackage[TS1,T1]{fontenc}
\usepackage[utf8]{luainputenc}
\usepackage[utf8]{inputenc}  
%\usepackage[T1]{fontenc}
\usepackage{amsfonts,amsmath,amssymb,latexsym} 
\usepackage{enumerate}
\usepackage{float}
\usepackage{graphicx}
\usepackage{caption}
\usepackage{subcaption}
\graphicspath{ {images/} }




\def\cQ{\mathcal{Q}}
\def\ve{\varepsilon}

\def\emptyset{\varnothing}
\def\ept{\varnothing}

% Include the listings-package
\usepackage{listings}             
\usepackage{color}

\definecolor{dkgreen}{rgb}{0,0.6,0}
\definecolor{gray}{rgb}{0.5,0.5,0.5}
\definecolor{mauve}{rgb}{0.58,0,0.82}

\lstset{frame=tb,
  language=Prolog,
  aboveskip=3mm,
  belowskip=3mm,
  showstringspaces=false,
  columns=flexible,
  basicstyle={\small\ttfamily},
  numbers=left,
  stepnumber=1,  
  numberstyle=\tiny\color{gray},
  keywordstyle=\color{blue},
  commentstyle=\color{dkgreen},
  stringstyle=\color{mauve},
  breaklines=true,
  breakatwhitespace=true,
  tabsize=3
}


\usepackage{tikz}
\usetikzlibrary{trees}



\usepackage[margin=1in]{geometry}
\usepackage{amsmath,amssymb}
\usepackage{multicol}

\def\code#1{\texttt{#1}}

\newcommand{\class}{IA}
\newcommand{\term}{1º semestre de 2020}
\newcommand{\examnum}{Lista 01}
\newcommand{\examdate}{}

\pointpoints{ponto}{pontos}

\pagestyle{head}
\firstpageheader{}{}{}
\runningheader{\class}{\examnum\ - Página \thepage\ de \numpages}{\examdate}
\runningheadrule


\begin{document}

\noindent
\begin{tabular*}{\textwidth}{l @{\extracolsep{\fill}} r @{\extracolsep{6pt}} l}
\textbf{\class} & \textbf{Nome:} & \makebox[2in]{\hrulefill}\\
\textbf{\term}  & \textbf{RA:}   & \makebox[2in]{\hrulefill}\\
\textbf{\examnum} &&\\
& Professor: & Vinicius Pereira
\end{tabular*}\\
\rule[2ex]{\textwidth}{2pt}

Esta lista contém \numpages\ páginas e \numquestions\ questões.\\


\noindent
\rule[2ex]{\textwidth}{2pt}


\begin{description}

\item[Espaço de estados]
Um espaço de estados é formado por uma 4-tupla $(N, A, I, O)$ onde
\begin{itemize}
\item $N$ é o conjunto de todos os estados possíveis.
\item $A$ é uma função que gera os filhos de um estado.
\item $I$ um subconjunto de estados que correspondem aos estados iniciais do problema
\item $O$ um subconjunto de estados que correspondem ao objetivo.
\end{itemize}
Deve ser fornecido regras claras, de como construir uma solução através de uma busca no espaço de estados

\item[Busca em largura]
A busca dá prioridade aos vértices mais próximos do vértice inicial

\item[Busca em profundidade]
A busca dá prioridade aos vértices mais distantes do vértice inicial

\end{description}

\vspace{3em}

\begin{questions}

\item Descreva formalmente um espaço de estados para o problema do fazendeiro.

\textbf{Definição do problema}: Um fazendeiro com seu lobo, sua cabra e seu repolho chega à margem de um rio que ele deseja atravessar. Há um barco que pode levar duas coisas, incluindo o remador (o fazendeiro), para o outro lado de cada vez. O lobo não pode ficar sozinho com a cabra e a cabra não pode ficar sozinha com o repolho. Uma solução é uma sequência de travessias pelo rio tal que todos terminem do outro lado.

Você deve descrever todos os 5 elementos definidos pelo espaço de estados e discutir as vantagens da busca em amplitude e profundidade para este problema.


\item Dê um exemplo para o problema do caixeiro viajante para o qual a estratégia gulosa do vizinho mais próximo não consiga encontrar uma solução ótima. Descreva formalmente um espaço de estados para o problema do caixeiro viajante. Sugira uma heurística para solucionar o problema e responda se a heurística sugerida encontra a solução ótima ou não.\\
\textbf{Definição do problema}: Uma instância do problema do caixeiro viajante é um grafo completo com peso nas arestas que satisfaz a desigualdade triangular. Uma solução válida é um ciclo que passe por todos os vértices, o custo deste ciclo é a soma do valor de cada uma de suas arestas. Uma solução ótima é o ciclo com menor custo.


%\item Descreva o espaço de busca, determine se a busca guiada por objetivo ou por dados seria melhor para os problemas a seguir e discuta as diferenças de uma busca por amplitude ou profundidade (se houver diferença):

%\begin{enumerate}
%
%\item Você encontrou uma pessoa que afirma ser a sua prima distante com um ancestral comum chamado João da Silva. Você deseja verificar a veracidade da afirmação.
%
%\item Você encontrou uma pessoa que afirma ser a sua prima distante. Ela não sabe o nome do ancestral comum, mas sabe que não está mais distante que 8 gerações. Você quer determinar este ancestral comum ou verificar que ele não existe.
%
%\end{enumerate}

\item Descreva o espaço de estado dos seguintes problemas:
\begin{enumerate}

\item Você tem que colorir um mapa plano com 4 cores de tal modo que não haja dois países vizinhos com a mesma cor.

%\item Um robô macaco com um metro de altura está em uma sala em que algumas bananas estão suspensas no teto a 2.5 metros de altura. A sala contém dois engradados empilháveis, móveis e escaláveis, com um metro de altura cada.

\item Você tem três jarros com capacidade para 12, 8 e 3 litros, você também tem uma torneira de água. É possível encher os jarros ou esvaziá-los passando a água de um para o outro ou derramando-a no chão. Você precisa conseguir exatamente um litro de água.

\end{enumerate}

\item Considere um espaço de estados onde o estado inicial tem valor 1 e o arco sucessor para o estado $n$ aponta para dois filhos com os número $2n$ e $2n+1$.
\begin{enumerate}

\item Desenhe a porção do grafo de busca com os estados que vão de 1 a 15.

\item Suponha que o estado objetivo seja o 11. Liste a ordem dos estados encontrados em uma busca em largura e em profundidade com limite 3.

%\item Repita os itens anteriores para uma busca guiada por objetivo.

\end{enumerate}

\end{questions}

\end{document}

