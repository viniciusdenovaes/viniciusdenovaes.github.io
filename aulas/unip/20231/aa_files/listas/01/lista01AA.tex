\documentclass[12pt]{article}

\usepackage[utf8]{inputenc}  
\usepackage[portuguese]{babel}   

\usepackage[ruled, vlined, algonl, portuguese]{algorithm2e}

\newcommand{\linesnumbered}{\LinesNumbered}
\newcommand{\dontprintsemicolon}{\DontPrintSemicolon}
\newcommand{\incmargin}{\IncMargin}
\newcommand{\decmargin}{\DecMargin}

\def\le{\leqslant}\def\leq{\le}
\def\ge{\geqslant}\def\geq{\ge}
\newcommand{\floor}[1]{\left\lfloor{#1}\right\rfloor}
\newcommand{\ceil}[1]{\left\lceil{#1}\right\rceil}
\newcommand{\pare}[1]{\left({#1}\right)}
\newcommand{\set}[1]{\left\{{#1}\right\}}
\newcommand{\range}[2]{\set{{#1},\dots,{#2}}}
\newcommand{\dist}{\mathrm{dist}}
\newcommand{\cluster}{\mathrm{cluster}}
\newcommand{\merge}{\mathrm{merge}}



\title{Lista 01 de Análise de Algoritmos}
\date{1º Período de 2022}
\author{Turma do 4º ano}


\begin{document} 

\maketitle

\vspace{3em}



\begin{enumerate}

\item Para cada problema abaixo, desenvolva um algoritmo, descreva o algoritmo (pode ser em alto nível, pseudo-código ou linguagem de programação) e diga qual é a complexidade do seu algoritmo na notação $O$.

OBS: O algoritmo pode usar operações como saber o tamanho do vetor e pegar um elemento do vetor da posição $i$ sem custo de cálculo.

OBS2: O seu algoritmo não precisa ter a menor complexidade conhecida.

\begin{enumerate}

\item Entrada: Um vetor de inteiros de tamanho $n$. Saída: a soma de todos elementos do vetor de entrada.

\item Entrada: Um vetor de reais de tamanho $n$. Saída: a média dos elementos do vetor de entrada.

\item Entrada: Um vetor de inteiros de tamanho $n$. Saída: o menor elemento do vetor.

\item Entrada: Um vetor de inteiros de tamanho $n$. Saída: o menor e o maior elemento do vetor.

\item Entrada: Um vetor de inteiros de tamanho $n$. Saída: o primeiro elemento do vetor.

\item Entrada: dois vetores de inteiros, cada um de tamanho $n$. Saída: um vetor cuja posição $i$ guarda a soma dos elementos da posição $i$ de cada vetor de entrada.

\item Entrada: Um vetor de inteiros de tamanho $n$. Saída: o vetor de entrada invertido.

\item Entrada: Duas matrizes quadradas de tamanho $n$. Saída: a soma das duas matrizes.

\item Entrada: Duas matrizes quadradas de tamanho $n$. Saída: a multiplicação das duas matrizes.

\item Entrada: Um vetor de inteiros de tamanho $n$. Saída: a mediana dos valores do vetor.

\item Entrada: Um vetor de inteiros de tamanho $n$ e um número inteiro. Saída: se o vetor contém este número.

\item Entrada: Um vetor de inteiros de tamanho $n$. Saída: se o vetor contém dois números iguais.

\item Entrada: dois vetores de inteiros, cada um de tamanho $n$. Saída: se os dois vetores têm um elemento em comum.

\item Entrada: Duas matrizes quadradas de tamanho $n$. Saída: se as duas matrizes têm um elemento em comum.



\end{enumerate}



\end{enumerate}

\end{document}

