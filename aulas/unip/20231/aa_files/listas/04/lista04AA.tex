\documentclass[12pt]{article}

\usepackage[brazil]{babel}   
\usepackage[utf8]{inputenc}  
%\usepackage[T1]{fontenc}
\usepackage{amsfonts,amsmath,amssymb,latexsym} 


\usepackage{tikz}
\usetikzlibrary{arrows,shapes,positioning,shadows,trees}




\def\cQ{\mathcal{Q}}
\def\ve{\varepsilon}
\newcommand\Tau{\mathcal{T}}






\def\emptyset{\varnothing}





\title{Lista 04 de Análise de Algoritmos}
\date{1º Período de 2023}
\author{Turma do 4º ano}

\begin{document} 

\maketitle

\begin{center}
    \large
    \textbf{Reduções Conhecidas\footnote{Apresentados em Algoritmos (Cormen)}}
    \vspace{0.4cm}
\end{center}

\begin{tikzpicture}[
  level 1/.style={sibling distance=40mm},
  edge from parent/.style={->,draw},
  >=latex,
  basic/.style  = {draw, font=\sffamily, rectangle, rounded corners=2pt, thin, align=center}]

% root of the the initial tree, level 1
\node[basic] {CIRCUIT-SAT}
  child { node[basic] {SAT}
    child { node[basic] {3-CNF-SAT}
      child {node[basic] (c1) {CLIQUE}
        child {node[basic] {VERTEX-COVER}
          child {node[basic] {HAM-CYCLE}
            child {node[basic] {TSP}}
          }
        }
      }
      child {node[basic] {SUBSET-SUM}
        child{node[basic] {SUBSET-PART}}      
      }
    }    
  };


\end{tikzpicture}

\break

\begin{center}
    \large
    \textbf{Descrições dos problemas}
    \vspace{0.4cm}
\end{center}


\begin{description}

\item[CIRCUIT-SAT]
Dado um circuito booleano, determinar se existe uma entrada que terá output verdadeiro

\item[SAT]
Dada uma expressão booleana, determinar se existe uma interpretação para as veaiáveis que satisfaça a expressão

\item[3-CNF-SAT]
Dada uma expressão booleana formada por conjunções de fórmulas formadas pela disjunção de 3 variáveis (ou sua negação). Determinar se existe uma interpretação para as veaiáveis que satisfaça a expressão


\item[CLIQUE]

Um clique em um grafo simples $G=(V,E)$ é um subconjunto de vértices $V'$ tal que todo vértice de $V'$ esteja ligado a todo vértice de $V'$. O tamanho deste clique é a quantidade de vértices em $V'$.
O problema é responder se existe um clique de tamanho igual a algum $k$ dado na entrada.

CLIQUE = $\{\langle G,V\rangle : G$ é um grafo com clique de tamanho igual a $k$ $\}$

\item[VERTEX-COVER]

O problema da cobertura de vértices é encontrar uma cobertura de vértices de tamanho $k$. 
Uma cobertura de vértices é um subconjunto $S$ de vértices, tal que, toda aresta do grafo seja incidente a pelo menos um dos vértices da cobertura $S$.
O tamanho de uma cobertura é uma quantidade de vértices que ela tem.

VERTEX-COVER = $\{\langle G,k\rangle : G$ é um grafo que tem uma cobertura de vértices de tamanho igual a $k$ $\}$

\item[HAM-CYCLE]

O problema do ciclo hamiltoniano é, em um grafo simples não direcionado, encontrar um ciclo que passe por cada vértices exatamente uma vez.

HAM-CYCLE = $\{\langle G\rangle : G$ tem um ciclo hamiltoniano$\}$


\item[TSP]

O problema do caixeiro viajante é, em um grafo simples, completo e com peso nas arestas, encontrar um cyclo que passe exatamente uma vez por todos os vértices e tenha custo menor ou igual a $k$.
O custo de um ciclo é a soma do custo das arestas pelas quais ele passa.

TSP = $\{\langle G, k\rangle : G$ tem um ciclo hamiltoniano de custo menor ou igual a $k$ $\}$

\item[SUBSET-SUM]

Temos um conjunto $S$ de números inteiros e um número inteiro $t$. O problema é responder se existe um subconjunto de $S$ tal que a soma de todos elementos resulte em $t$.

SUBSET-SUM = $\{\langle S,t\rangle$ : existe um subconjunto $S'\subseteq S$ tal que $t = \sum_{s\in S'}s$ $\}$

\item[SUBSET-PART]

O problema da partição de um conjunto é, dado um conjunto, dizer se há como particionar em dois subconjunto tal que a soma seja igual.

SUBSET-PART = $\{\langle S,t\rangle$ : existe uma partição de $S$, dada por dois conjuntos $A$ e $\bar{A}$, tal que $\sum_{a\in A}a$ $ = \sum_{a\in \bar{A}} a$ $\}$

\item[MULTI-SUBSET-SUM]

Temos um multiconjunto $S$ de números inteiros e um número inteiro $t$. O problema é responder se existe um multisubconjunto de $S$ tal que a soma de todos elementos resulte em $t$.

MULTI-SUBSET-SUM = $\{\langle S,t\rangle$ : existe um multisubconjunto $S'\subseteq S$ tal que $t = \sum_{s\in S'}s$ $\}$

\item[MULTI-SUBSET-PART]

O problema da partição de um multiconjunto é, dado um conjunto, dizer se há como particionar em dois multisubconjunto tal que a soma seja igual.

MULTI-SUBSET-PART = $\{\langle S,t\rangle$ : existe uma partição de $S$, dada por dois conjuntos $A$ e $\bar{A}$, tal que $\sum_{a\in A}a$ $ = \sum_{a\in \bar{A}} a$ $\}$

\item[Definições]:

\begin{description}

\item[multiconjunto] é um conjunto que permite a repetição de elementos

\item[multisubconjunto] é um subconjunto, de um multisubconjunto, que permite a repetição de elementos.

\end{description}


\end{description}


\begin{description}

\item[Provar que um problema é NP:] dada uma possível solução, descrever um algoritmo polinomial para verificar se esta é uma solução válida para o problema

\item[Provar que um problema é P:] exibir um algoritmo polinomial para o problema

\item[Supondo que P $\neq$ NP:] para esta lista estamos supondo que \textbf{não} existe algoritmo polinomial para os problemas apresentados no início desta lista

\end{description}







\begin{enumerate}

\item Diga quais dos seguintes problemas pertencem à classe P, quais pertencem à NP, e quais pertencem às duas classes. Justifique a sua resposta.

\begin{enumerate}

\item Ordenar um vetor de tamanho $n$

\item Multiplicar duas matrizes de tamanho $n\times n$

\item Encontrar o caminho mínimo entre dois vértices de um grafo completo com custo nas arestas

\item Encontrar um caminho com custo menor que um dado $k$, entre dois vértices de um grafo completo, com custo nas arestas

\item Dado um grafo simples, encontrar um subconjunto $S$, de tamanho $k$, de vértices neste grafo, tal que todos os vértices deste subconjunto $S$ esteja ligado a todos os vértices do próprio subconjunto $S$

\item Dado um grafo simples, encontrar um subconjunto $S$, de tamanho $k$, de vértices neste grafo, tal que todos os vértices deste subconjunto $S$ esteja ligado a todos os vértices do conjunto $V$ de vértices do grafo

\end{enumerate}


\item Mostre que os seguintes problemas são NP

\begin{enumerate}

\item TSP

\item CLIQUE

\item VERTEX-COVER

\item HAM-CYCLE

\item SUBSET-PART

\end{enumerate}




\break


\item Mostre que os seguintes problemas são NP-Completos

\begin{enumerate}

\item Problema da rota de veículos: Dado um grafo $G = (V,E)$ simples, completo, com custo nas arestas. Dado um subconjunto $T\subseteq V$, dado um número real $k$. Encontrar um conjunto de ciclos, tal que, cada ciclo passe por exatamente um vértice de $T$; todo vértice de $V$ esteja em um ciclo; e a soma dos custos de todos os ciclos seja menor ou igual a $k$. (Considere que vértices sozinhos formam um ciclo)

\item Problema do conjunto independente: Dado um grafo $G = (V,E)$ simples, encontrar um subconjunto $S\subseteq V$ tal que cada vértice de $S$ não seja vizinho a nenhum vértice de $S$. (Dica: reduza o problema do CLIQUE para este)

\end{enumerate}

\item Bonie e Clyde roubaram um banco e agora querem dividir os ganhos, mas será que o que eles roubaram pode ser dividido igualmente? 
Verifique em quais das situações é possível encontrar um algoritmo polinomial para dividir os ganhos.
Justifique a sua resposta.

\begin{enumerate}

\item Todas as notas roubadas tem o mesmo valor.

\item Todo material roubado são cheques, endereçado a eles, com valores diferentes um do outro.

\item A mesma situação do item anterior, mas desta vez eles aceitam que a divisão tenha uma difereça de 10 reais.

\end{enumerate}





\end{enumerate}

\end{document}

