\documentclass[12pt]{article}

\usepackage[utf8]{inputenc}  
\usepackage[portuguese]{babel}   

\usepackage[ruled, vlined, algonl, portuguese]{algorithm2e}

\usepackage{enumerate}

\newcommand{\linesnumbered}{\LinesNumbered}
\newcommand{\dontprintsemicolon}{\DontPrintSemicolon}
\newcommand{\incmargin}{\IncMargin}
\newcommand{\decmargin}{\DecMargin}

\def\le{\leqslant}\def\leq{\le}
\def\ge{\geqslant}\def\geq{\ge}
\newcommand{\floor}[1]{\left\lfloor{#1}\right\rfloor}
\newcommand{\ceil}[1]{\left\lceil{#1}\right\rceil}
\newcommand{\pare}[1]{\left({#1}\right)}
\newcommand{\set}[1]{\left\{{#1}\right\}}
\newcommand{\range}[2]{\set{{#1},\dots,{#2}}}
\newcommand{\dist}{\mathrm{dist}}
\newcommand{\cluster}{\mathrm{cluster}}
\newcommand{\merge}{\mathrm{merge}}



\title{Lista 01 de Análise de Algoritmos}
\date{1º Período de 2022}
\author{Turma do 4º ano}

\begin{document}

\maketitle

\begin{description}

\item[Espaço de estados]
Um espaço de estados é formado por uma 4-tupla $(N, A, I, O)$ onde
\begin{itemize}
\item $N$ é o conjunto de todos os estados possíveis.
\item $A$ é uma função que gera os filhos de um estado.
\item $I$ um subconjunto de estados que correspondem aos estados iniciais do problema
\item $O$ um subconjunto de estados que correspondem ao objetivo.
\end{itemize}
Deve ser fornecido regras claras, de como construir uma solução através de uma busca no espaço de estados

\item[Busca em largura]
A busca dá prioridade aos vértices mais próximos do vértice inicial

\item[Busca em profundidade]
A busca dá prioridade aos vértices mais distantes do vértice inicial

\end{description}

\break

\begin{enumerate}

\item Descreva formalmente um espaço de estados para o problema do fazendeiro.

\textbf{Definição do problema}: Um fazendeiro com seu lobo, sua cabra e seu repolho chega à margem de um rio que ele deseja atravessar. Há um barco que pode levar duas coisas, incluindo o remador (o fazendeiro), para o outro lado de cada vez. O lobo não pode ficar sozinho com a cabra e a cabra não pode ficar sozinha com o repolho. Uma solução é uma sequência de travessias pelo rio tal que todos terminem do outro lado.

Você deve descrever todos os 5 elementos definidos pelo espaço de estados e discutir as vantagens da busca em amplitude e profundidade para este problema.



\break





\item Considere o Problema do Caixeiro Viajante (TSP): Um vendedor sai de sua casa planejando passar em um conjunto de casas escritas em uma lista. Qual percurso este vendedor deve fazer para andar o menor tempo possível?

\textbf{Definição formal do problema}: Uma instância do problema do caixeiro viajante é um grafo completo com peso nas arestas que satisfaz a desigualdade triangular. Uma solução válida é um ciclo que passe por todos os vértices, o custo deste ciclo é a soma do valor de cada uma de suas arestas. Uma solução ótima é o ciclo com menor custo.

\begin{enumerate}[a)]

\item Considere a estratégia do vendedor sempre pegar o caminho para a casa que falta mais próxima de onde ele está parado. Dê um exemplo para o problema do caixeiro viajante para no qual esta estratégia míope não consiga encontrar uma solução ótima. 

\item Descreva formalmente um espaço de estados para o problema do caixeiro viajante. 

\item Escreva uma heurística para solucionar o problema e responda se a heurística sugerida encontra a solução ótima ou não.\\
\end{enumerate}





%\item Descreva o espaço de busca, determine se a busca guiada por objetivo ou por dados seria melhor para os problemas a seguir e discuta as diferenças de uma busca por amplitude ou profundidade (se houver diferença):

%\begin{enumerate}
%
%\item Você encontrou uma pessoa que afirma ser a sua prima distante com um ancestral comum chamado João da Silva. Você deseja verificar a veracidade da afirmação.
%
%\item Você encontrou uma pessoa que afirma ser a sua prima distante. Ela não sabe o nome do ancestral comum, mas sabe que não está mais distante que 8 gerações. Você quer determinar este ancestral comum ou verificar que ele não existe.
%
%\end{enumerate}





\item Descreva o espaço de estado dos seguintes problemas:
\begin{enumerate}[a)]

\item Você tem que colorir um mapa plano com 4 cores de tal modo que não haja dois países vizinhos com a mesma cor.

%\item Um robô macaco com um metro de altura está em uma sala em que algumas bananas estão suspensas no teto a 2.5 metros de altura. A sala contém dois engradados empilháveis, móveis e escaláveis, com um metro de altura cada.

\item Você tem três jarros com capacidade para 12, 8 e 3 litros, você também tem uma torneira de água. É possível encher os jarros ou esvaziá-los passando a água de um para o outro ou derramando-a no chão. Você precisa conseguir exatamente um litro de água.

\end{enumerate}



\break

\item Considere um espaço de estados onde o estado inicial tem valor 1 e o arco sucessor para o estado $n$ aponta para dois filhos com os número $2n$ e $2n+1$.
\begin{enumerate}[a)]

\item Desenhe a porção do grafo de busca com os estados que vão de 1 a 15.

\item Suponha que o estado objetivo seja o 11. Liste a ordem dos estados encontrados em:

\begin{itemize}

\item uma busca em largura.

\item em profundidade com limite de 3 níveis.

\end{itemize}
%\item Repita os itens anteriores para uma busca guiada por objetivo.

\end{enumerate}





\break


\item[\textbf{Extra}] Um carro guiado por uma inteligência artificial, deve sair do repouso, correr por uma ponte reta até chegar em um penhasco, onde ele deve dar um salto até chegar do outro lado do penhasco, onde ele deve desacelerar até voltar ao repouso antes de bater em um muro. 

A situação descrita é simulada por um computador, onde temos as seguintes regras:

\begin{itemize}

\item Em cada instante de tempo $t = 0, 1, 2, \dots$ (em segundos) o carro deve decidir entre \textbf{M}anter a velocidade, \textbf{A}celerar, ou \textbf{D}esacelerar. 

\item Se o carro está a uma velocidade $v_i$ no instante $t_i$ e no ponto $p_i$. 

\begin{itemize}

\item Caso o carro decida \textbf{M}anter a velocidade: no instante $t_i+1$ o carro estará com uma velocidade $v_i$, no ponto $p_i+v_i$. 

\item Caso o carro decida \textbf{A}celerar: no instante $t_i+1$ o carro estará com uma velocidade $v_i+1$, no ponto $p_i+v_i+1$. 

\item Caso o carro decida \textbf{D}esacelerar: no instante $t_i+1$ o carro estará com uma velocidade $v_i-1$, no ponto $p_i+v_i-1$. 

\end{itemize}

\item O carro inicia no ponto $0$ e com velocidade $0$. 

\item O objetivo é fazer com que o carro chegue até o ponto $D$ com a velocidade mínima para que ele salte o penhasco de comprimento $H$, para isto, no ponto $D$ ele deve estar com uma velocidade maior ou igual a $H$ e decidir saltar.

\end{itemize}

Descreva um espaço de busca para o problema e desenhe a árvore de  busca para uma ``busca guiada por objetivo'' para os valores $H = 4$ e $D = 11$. 

\textbf{Busca Guiada Por Objetivo} é percorrer a árvore a partir de um estado objetivo, buscando por algum estado inicial.

\textbf{OBS:} Represente a sua árvore em formato de uma matriz, onde cada coluna é uma distância e cada linha é uma velocidade

\break















\end{enumerate}

\end{document}

