\documentclass[12pt]{article}

\usepackage[brazil]{babel}   
\usepackage[utf8]{inputenc}  
%\usepackage[T1]{fontenc}
\usepackage{amsfonts,amsmath,amssymb,latexsym} 

\def\cQ{\mathcal{Q}}
\def\ve{\varepsilon}
\newcommand\Tau{\mathcal{T}}

\def\emptyset{\varnothing}



\title{Lista 03 de Análise de Algoritmos}
\date{1º Período de 2022}
\author{Turma do 4º ano}

\begin{document} 

\maketitle

\begin{description}

\item[Notação $O$]: 

$f(n) = O(g(n))$ se existe $n_0>0$ e $c>0$ talque $f(n) \leq c\cdot g(n)$ para todo $n>n_0$.

\end{description}

\vspace{3em}



\begin{enumerate}

\item Se temos $f(n) = O(g(n))$, quais das seguintes alternativas são verdadeiras?

\begin{itemize}

\item[( )]  $f(n) = 2n + 1$, $g(n) = n^2$.

\item[( )]  $f(n) = 2n + 1$, $g(n) = n$.

\item[( )]  $f(n) = 2n + 1$, $g(n) = 1$.

\item[( )]  $f(n) = \lg n + n$, $g(n) = \lg n$.

\item[( )]  $f(n) = \lg n + n$, $g(n) = n$.

\item[( )]  $f(n) = \lg n + 1$, $g(n) = \lg n$.

\item[( )]  $f(n) = \lg n + 1$, $g(n) = n$.

\item[( )]  $f(n) = 4n^2 + n + 8$, $g(n) = n^3$.

\item[( )]  $f(n) = 4n^2 + n + 8$, $g(n) = n^2$.

\item[( )]  $f(n) = 4n^2 + n + 8$, $g(n) = n$.

\item[( )]  $f(n) = 4n^2 + n + 8$, $g(n) = 1$.

\item[( )]  $f(n) = 8$, $g(n) = 1$.

\item[( )]  $f(n) = n\lg n$, $g(n) = n^2$.

\item[( )]  $f(n) = n\lg n$, $g(n) = \lg n$.

\end{itemize}


\end{enumerate}

\end{document}

