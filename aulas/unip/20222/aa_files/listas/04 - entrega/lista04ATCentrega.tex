\documentclass[12pt]{article}

\usepackage[brazil]{babel}   
\usepackage[utf8]{inputenc}  
%\usepackage[T1]{fontenc}
\usepackage{amsfonts,amsmath,amssymb,latexsym} 


\usepackage{tikz}
\usetikzlibrary{arrows,shapes,positioning,shadows,trees}




\def\cQ{\mathcal{Q}}
\def\ve{\varepsilon}
\newcommand\Tau{\mathcal{T}}






\def\emptyset{\varnothing}





\title{Lista 04 de Aspectos Teóricos da Computação
\\PARA ENTREGAR
}
\date{1º Período de 2018}
\author{Turma do 4º ano}

\begin{document} 

\maketitle


\begin{enumerate}

\item Diga quais dos seguintes problemas pertencem à classe P, quais pertencem à NP, e quais pertencem às duas classes. Justifique a sua resposta.

\begin{enumerate}

\item Dado um grafo simples, encontrar um subconjunto $S$, de tamanho $k$, de vértices neste grafo, tal que todos os vértices deste subconjunto $S$ esteja ligado a todos os vértices do próprio subconjunto $S$

\item Dado um grafo simples, encontrar um subconjunto $S$, de tamanho $k$, de vértices neste grafo, tal que todos os vértices deste subconjunto $S$ esteja ligado a todos os vértices do conjunto $V$ de vértices do grafo

\end{enumerate}


\item Mostre que os seguintes problemas são NP

\begin{enumerate}

\item CLIQUE

\item SUBSET-PART

\end{enumerate}

\item Mostre que os seguintes problemas são NP-Completos

\begin{enumerate}

\item Problema da rota de veículos: Dado um grafo $G = (V,E)$ simples, completo, com custo nas arestas. Dado um subconjunto $T\subseteq V$, dado um número real $k$. Encontrar um conjunto de ciclos, tal que, cada ciclo passe por exatamente um vértice de $T$; todo vértice de $V$ esteja em um ciclo; e a soma dos custos de todos os ciclos seja menor ou igual a $k$. (Considere que vértices sozinhos formam um ciclo)

\item Problema do conjunto independente: Dado um grafo $G = (V,E)$ simples, encontrar um subconjunto $S\subseteq V$ tal que cada vértice de $S$ não seja vizinho a nenhum vértice de $S$. (Dica: reduza o problema do CLIQUE para este)

\end{enumerate}

\item Bonie e Clyde roubaram um banco e agora querem dividir os ganhos, mas será que o que eles roubaram pode ser dividido igualmente? 
Verifique em quais das situações é possível encontrar um algoritmo polinomial para dividir os ganhos.
Justifique a sua resposta.

\begin{enumerate}

\item A mesma situação do item anterior, mas desta vez eles aceitam que a divisão tenha uma difereça de 10 reais.

\item Todo material roubado são cheques, endereçado a eles, com valores diferentes um do outro.

\item A mesma situação do item anterior, mas desta vez eles aceitam que a divisão tenha uma difereça de 10 reais.

\end{enumerate}





\end{enumerate}

\end{document}

