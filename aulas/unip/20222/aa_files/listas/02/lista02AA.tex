\documentclass[12pt]{article}

\usepackage[utf8]{inputenc}  
\usepackage[portuguese]{babel}   

\usepackage[ruled, vlined, algonl, portuguese]{algorithm2e}

\newcommand{\linesnumbered}{\LinesNumbered}
\newcommand{\dontprintsemicolon}{\DontPrintSemicolon}
\newcommand{\incmargin}{\IncMargin}
\newcommand{\decmargin}{\DecMargin}

\def\le{\leqslant}\def\leq{\le}
\def\ge{\geqslant}\def\geq{\ge}
\newcommand{\floor}[1]{\left\lfloor{#1}\right\rfloor}
\newcommand{\ceil}[1]{\left\lceil{#1}\right\rceil}
\newcommand{\pare}[1]{\left({#1}\right)}
\newcommand{\set}[1]{\left\{{#1}\right\}}
\newcommand{\range}[2]{\set{{#1},\dots,{#2}}}
\newcommand{\dist}{\mathrm{dist}}
\newcommand{\cluster}{\mathrm{cluster}}
\newcommand{\merge}{\mathrm{merge}}




\title{Lista 02 de Análise de Algoritmos}
\date{1º Período de 2022}
\author{Turma do 4º ano}




\begin{document}


\maketitle


\vspace{3em}

Em aula foram dados dois algoritmos de ordenação, o Insertion Sort e o Merge Sort. 

O Insertion Sort, um algoritmo mais simples, supõe que o vetor está ordenado da posição inicial até uma certa posição $n$. O procedimento então pega o elemento da próxima posição e coloca em seu lugar devido na parte ordenada do vetor, deixando o vetor ordenado até a posição $n+1$.
Repetindo este procedimento com $n$ variando de $1$ até o tamanho do vetor, o resultado é o vetor todo ordenado.

O Merge Sorte usa a técnica de divisão-e-conquista, o algoritmo divide o vetor ao meio, em dois vetores de tamanho igual, ordena cada um deles, depois junta os dois, de forma ordenada, em um terceiro vetor.

Os dois algoritmos são mostrados a seguir, na forma de um pseudo-código (O algoritmo Merge é usado pelo Merge-Sort).
Encontre o tempo de execução de pior caso do Insertion-Sort e do Merge-Sort, em termos da notação $O$, em relação ao tamanho do vetor $A$. \textbf{Escreva o raciocínio que você usou.}


%\incmargin{1em} 
\linesnumbered \dontprintsemicolon
\begin{algorithm}[ht!]{
  \footnotesize
  \caption{Merge-sort$(A,p,r)$\label{alg:merge-sort}}

  \Entrada{Um vetor $A$, a posição do início do vetor $p$ e a posição final $r$.}
  \Saida{O vetor $A$ ordenado da posição $p$ até a posição $r$.}
  
  \Se{$p<r$}{
    $q = \lfloor (p+r)/2\rfloor$\;
  }
  Merge-Sort($A$, $p$, $q$)\;
  Merge-Sort($A$, $q+1$, $r$)\;
  Merge($A$, $p$, $q$, $r$)\;
}
\end{algorithm}


%\incmargin{1em} 
\linesnumbered \dontprintsemicolon
\begin{algorithm}[ht!]{
  \footnotesize
  \caption{Merge$(A,p,q,r)$\label{alg:merge}}

  \Entrada{Um vetor $A$, a posição do início do vetor $p$, a posição do meio $q$ e a posição final $r$.}
  \Saida{O vetor $A$ ordenado da posição $p$ até a posição $r$.}
  
  $n_1 = q-p+1$\;
  $n_2 = r-q$\;
  Seja $L[1..n_1+1]$ e $R[1..n_2+1]$ novos vetores\;
  \Para{$i=1$ \Ate $n_1$}{
    $L[i] = A[p+i-1]$\;
  }
  \Para{$j=1$ \Ate $n_2$}{
    $R[j] = A[q+j]$\;
  }
  $L[n_1+1] = \infty$\;
  $R[n_2+1] = \infty$\;
  $i=1$\;
  $j=1$\;
  \Para{$k=p$ \Ate $r$}{
    \Se{$L[i] <= R[j]$}{
      $A[k] = L[i]$\;
      $i=i+1$\;
    }
    \Senao{
      $A[k] = R[j]$\;
      $j=j+1$\;
    }
  }
  
}
\end{algorithm}


%\incmargin{1em} 
\linesnumbered \dontprintsemicolon
\begin{algorithm}[ht!]{
  \footnotesize
  \caption{Insertion-Sort$(A)$\label{alg:insertion-sort}}

  \Entrada{Um vetor $A$ de tamanho $n$.}
  \Saida{O vetor $A$ ordenado.}
  
  \Para{$j=2$ \Ate $n$}{
    $key = A[j]$\;
    $i=j-1$\;
    \Enqto{$i>0$ \textnormal{e} $A[i]>key$}{
      $A[i+1] = A[i]$\;
      $i=i-1$\;
    }
    $A[i+1] = key$\;
  }
}
\end{algorithm}


\end{document}
