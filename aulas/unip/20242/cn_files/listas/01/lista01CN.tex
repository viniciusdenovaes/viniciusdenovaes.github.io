\documentclass[12pt]{article}

\usepackage[utf8]{inputenc}  
\usepackage[portuguese]{babel}   

\usepackage[ruled, vlined, algonl, portuguese]{algorithm2e}

\newcommand{\linesnumbered}{\LinesNumbered}
\newcommand{\dontprintsemicolon}{\DontPrintSemicolon}
\newcommand{\incmargin}{\IncMargin}
\newcommand{\decmargin}{\DecMargin}

\usepackage{mathtools}

\usepackage{amssymb}
\def\le{\leqslant}\def\leq{\le}
\def\ge{\geqslant}\def\geq{\ge}
\newcommand{\floor}[1]{\left\lfloor{#1}\right\rfloor}
\newcommand{\ceil}[1]{\left\lceil{#1}\right\rceil}
\newcommand{\pare}[1]{\left({#1}\right)}
\newcommand{\set}[1]{\left\{{#1}\right\}}
\newcommand{\range}[2]{\set{{#1},\dots,{#2}}}
\newcommand{\dist}{\mathrm{dist}}
\newcommand{\cluster}{\mathrm{cluster}}
\newcommand{\merge}{\mathrm{merge}}



\title{Lista de Cálculo Numérico}
\date{2º Período de 2023}



\begin{document}


\maketitle


\vspace{3em}

\section{Medida de Erro}

Ao usar o computador para fazer cálculos numéricos sempre precisamos esperar que o resultado não seja exato, pois não é possível para o computador representar uma infinidade de números. Além disso, para fins práticos, nem sempre precisamos do resultado exato, uma aproximação já basta, e pode nos economizar tempo, poder de processamento e memória. Mais ainda, se tivermos trabalhando com resultados coletados no mundo real, e tentando fazer uma modelagem matemática que explique estes resultado, pode não ser possível (ou desejável) fazer uma modelagem matemática sem considerar uma função de erro.

Então precisamos definir uma função para medir o erro do nosso resultado (ou da modelagem matemática).

Para esta lista vamos usar a seguinte função: 
Para os pontos de amostra $(x_i, y_i)$, $i = 1, ..., n$, e a função matemática $f(x)$, o erro associado a $f(x)$ na amostra será
\[\sum_{i=1}^{n}\frac{(f(x_i) - y_i)^2}{n}\]

\textbf{Exercício}: Dada a função $2x^2 + -4x + -30$ e os pontos $(-2, -12)$, $(-2, -16)$, $(0, -29)$, $(0, -31)$, $(1, -29)$ e $(1, -35)$; encontre o erro de cada ponto em relação à função.



\break





\section{Resolução de Sistema de Equações Lineares}

Uma das maneiras de resolver um sistema de equações lineares é transformar o sistema em um \textit{sistema triangular superior}: um sistema onde os termos na parte inferior da diagona do sistema são iguais a zero.

Para isto podemos fazer as seguintes operações no sistema:
\begin{itemize}
\item trocar duas equações
\item multiplicar uma equação por uma constante
\item adicionar um multiplo de uma equação a uma outra equação
\end{itemize}

\textbf{Exercício: }

Transforme o sistema abaixo em um sistema triangular superior e depois resolva o sistema

\begin{tabular}{ccccccc}
$  6x$ & $-$ & $ 2y$ & $+$ & $10z$ & $=$ & $-152$\\
$ -2x$ & $+$ & $10y$ & $-$ & $14z$ & $=$ & $-8$\\
$ 10x$ & $-$ & $14y$ & $+$ & $34z$ & $=$ & $-176$
\end{tabular}





\break





\section{Regressão Linear (pelo método dos mínimos quadrados)}

Dado um conjunto de amostras $(x_i, y_i)$, para $i = 1, 2, ..., m$, e um conjuntos de funções $g_1(x), g_2(x), ..., g_n(x)$, queremos encontra os valores $\alpha_1, \alpha_2, ..., \alpha_n$ tal que o valor do erro entre os dados de amostra e a função $f(x) = \alpha_1g_1(x) + \alpha_2g_2(x) + \cdots + \alpha_ng_n(x)$ seja mínimo. Sendo este erro referente à medida de erro introduzida no início da lista.

Vimos em aula que existe uma fórmula para encontrar os valores de $\alpha_j$, que são os valores que satisfazem a equação.

\begin{tabular}{ccccccccc}
   $a_{11}\alpha_1$ & $+$ & $a_{12}\alpha_2$ & $+$ & $\cdots$ & + & $a_{1n}\alpha_n$ & $=$ & $b_1$ \\
   $a_{21}\alpha_1$ & $+$ & $a_{22}\alpha_2$ & $+$ & $\cdots$ & + & $a_{2n}\alpha_n$ & $=$ & $b_2$ \\
   $\vdots$         & $+$ & $\vdots$         & $+$ & $\vdots$ & + &         $\vdots$ & $=$ &$\vdots$ \\
   $a_{n1}\alpha_1$ & $+$ & $a_{n2}\alpha_2$ & $+$ & $\cdots$ & + & $a_{nn}\alpha_n$ & $=$ & $b_n$ 
\end{tabular}

onde 


\[a_{ij} = \sum_{k=1}^{m}g_i(x_k)g_j(x_k) = a_{ji}\]
\[b_i = \sum_{k=1}^{m}y_kg_i(x_k)\]


\textbf{Exercício}: Dados os pontos $(-2, -12)$, $(-2, -16)$, $(0, -29)$, $(0, -31)$, $(1, -29)$ e $(1, -35)$ econtre (i) uma função linear contante, (ii) uma função linear e (iii) uma função quadrática que minimize o erro entre os pontos  amostrais e as funções encontradas


\end{document}
