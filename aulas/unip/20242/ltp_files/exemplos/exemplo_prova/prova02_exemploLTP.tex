\documentclass[12pt]{exam}

\usepackage[brazil]{babel}   
\usepackage[TS1,T1]{fontenc}
\usepackage[utf8]{luainputenc}
\usepackage[utf8]{inputenc}  
%\usepackage[T1]{fontenc}
\usepackage{amsfonts,amsmath,amssymb,latexsym} 
\usepackage{enumerate}
\usepackage{float}
\usepackage{graphicx}
\usepackage{caption}
\usepackage{subcaption}
\graphicspath{ {images/} }




\def\cQ{\mathcal{Q}}
\def\ve{\varepsilon}

\def\emptyset{\varnothing}
\def\ept{\varnothing}

% Include the listings-package
\usepackage{listings}             
\usepackage{color}

\definecolor{dkgreen}{rgb}{0,0.6,0}
\definecolor{gray}{rgb}{0.5,0.5,0.5}
\definecolor{mauve}{rgb}{0.58,0,0.82}

\lstset{frame=tb,
  language=Prolog,
  aboveskip=3mm,
  belowskip=3mm,
  showstringspaces=false,
  columns=flexible,
  basicstyle={\small\ttfamily},
  numbers=left,
  stepnumber=1,  
  numberstyle=\tiny\color{gray},
  keywordstyle=\color{blue},
  commentstyle=\color{dkgreen},
  stringstyle=\color{mauve},
  breaklines=true,
  breakatwhitespace=true,
  tabsize=3
}


\usepackage{tikz}
\usetikzlibrary{trees}



\usepackage[margin=1in]{geometry}
\usepackage{amsmath,amssymb}
\usepackage{multicol}

\def\code#1{\texttt{#1}}

\newcommand{\class}{LTP}
\newcommand{\term}{2º semestre de 2024}
\newcommand{\examnum}{Prova Exemplo}
\newcommand{\examdate}{?/?/2024}
\newcommand{\timelimit}{$\thickapprox$75 Minutos}

\pointpoints{ponto}{pontos}

\pagestyle{head}
\firstpageheader{}{}{}
\runningheader{\class}{\examnum\ - Página \thepage\ de \numpages}{\examdate}
\runningheadrule


\begin{document}

\noindent
\begin{tabular*}{\textwidth}{l @{\extracolsep{\fill}} r @{\extracolsep{6pt}} l}
\textbf{\class} & \textbf{Nome:} & \makebox[2in]{\hrulefill}\\
\textbf{\term}  & \textbf{RA:}   & \makebox[2in]{\hrulefill}\\
\textbf{\examnum} & \textbf{Turma:}   & \makebox[2in]{\hrulefill}\\
\textbf{\examdate} &&\\
\textbf{Tempo limite: \timelimit} & Professor: & Vinicius Pereira \\
                                  &              &
\end{tabular*}\\
\rule[2ex]{\textwidth}{2pt}

AVISOS:

Ao escrever o seu código você não precisa se preocupar com os comandos de \code{include}

Ao escrever o seu código, ou o resultado de algum código, você pode ignorar as quebras de linha.

Esta prova contém \numpages\ páginas e \numquestions\ questões.\\


\begin{center}
Folha de respostas\\
\begin{tabular}{ c|c|c|c|c|c|c|c| }
 \hline
  & A & B & C & D & E & Nota & Máxima \\ \hline\hline
 \pgfsetfillopacity{1.0}1 \pgfsetfillopacity{0.3} & A & B & C & D & E & & 1 \\[-1.6ex]
 \hline\noalign{\vspace{\dimexpr 1.6ex-\doublerulesep}}
 \hline
 \pgfsetfillopacity{1.0}2 \pgfsetfillopacity{0.3} & A & B & C & D & E & & 1 \\ \hline
 \pgfsetfillopacity{1.0}3 \pgfsetfillopacity{0.3} & A & B & C & D & E & & 1 \\[-1.6ex]
 \hline\noalign{\vspace{\dimexpr 1.6ex-\doublerulesep}}
 \hline
 \pgfsetfillopacity{1.0}4 \pgfsetfillopacity{0.3} & A & B & C & D & E & & 1 \\ \hline
 \pgfsetfillopacity{1.0}5 \pgfsetfillopacity{0.3} & A & B & C & D & E & & 1 \\ \hline
 \pgfsetfillopacity{1.0}6 \pgfsetfillopacity{0.3} & A & B & C & D & E & & 1 \\ \hline
 \pgfsetfillopacity{1.0}7 \pgfsetfillopacity{0.3} & A & B & C & D & E & & 1 \\ \hline
 \pgfsetfillopacity{1.0}8 \pgfsetfillopacity{0.3} & A & B & C & D & E & & 1 \\ \hline
 \pgfsetfillopacity{1.0}9 \pgfsetfillopacity{0.3} & A & B & C & D & E & & 1 \\ \hline
 \pgfsetfillopacity{1.0}10\pgfsetfillopacity{0.3} & A & B & C & D & E & & 1 \\ \hline
 \end{tabular}
\end{center}
\pgfsetfillopacity{1.0}





\begin{questions}



% n_pares.c  pares.c soma.c hello_world.c  par.c      primo.c  sequencia.c








\question[1] Escreva um programa que escreva na tela a frase ``\code{Hello World}''

\makeemptybox{\stretch{1}}

\break







\question[1] Considere o seguinte programa escreva a saída

\lstinputlisting[language=C]{q_sem_ref.c}


\question[1] Considere o seguinte programa escreva a saída

\lstinputlisting[language=C]{q_com_ref.c}


\question[1] Considere o seguinte programa escreva a saída

\lstinputlisting[language=C]{q_f_array.c}

\question[1] Faça uma função que recebe dois inteiros e retorna o maior deles.

\question[1] Faça uma função que recebe duas medidas da largura e altura de um retângulo e retorne a área do retângulo.

\question[1] Faça uma função que recebe um número inteiro e retorna `1` caso este número seja *par* e `0` caso não seja *par*.

\question[1] Considere o seguinte programa escreva a saída para a entradas 4, 5, 6, 7
\lstinputlisting[language=C]{q_prime.c}

\question[1] Faça um programa que leia 5 valores inteiros entrados pelo usuário e imprima **o dobro** destes valores **na ordem inversa**.

\question[1] Faça um programa que leia dois arrays A e B de tamanho 6 de números reais entrados pelo usuário, faça um terceiro array C em que cada elemento de C é o elemento de A menos o elemento de B na mesma posição. Exiba todo o array C.


\end{questions}


\end{document}
