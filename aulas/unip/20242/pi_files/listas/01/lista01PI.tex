\documentclass[12pt]{article}

\usepackage[utf8]{inputenc}  
\usepackage[portuguese]{babel}   

\usepackage[ruled, vlined, algonl, portuguese]{algorithm2e}

\newcommand{\linesnumbered}{\LinesNumbered}
\newcommand{\dontprintsemicolon}{\DontPrintSemicolon}
\newcommand{\incmargin}{\IncMargin}
\newcommand{\decmargin}{\DecMargin}

\usepackage{mathtools}

\usepackage{amssymb}
\usepackage{amsmath}
\usepackage[shortlabels]{enumitem}

\def\le{\leqslant}\def\leq{\le}
\def\ge{\geqslant}\def\geq{\ge}
\newcommand{\floor}[1]{\left\lfloor{#1}\right\rfloor}
\newcommand{\ceil}[1]{\left\lceil{#1}\right\rceil}
\newcommand{\pare}[1]{\left({#1}\right)}
\newcommand{\set}[1]{\left\{{#1}\right\}}
\newcommand{\range}[2]{\set{{#1},\dots,{#2}}}
\newcommand{\dist}{\mathrm{dist}}
\newcommand{\cluster}{\mathrm{cluster}}
\newcommand{\merge}{\mathrm{merge}}



\title{Lista de Processamento de Imagens}
\date{}



\begin{document}


\maketitle


\vspace{3em}

\section{Tons de Cinza}

Tradicionalmente uma imagem digital é representada por uma matriz, em que cada célula contêm uma tupla de valores representando os canais $r$, $g$ e $b$.

Uma imagem em \textit{tons de cinza} é uma imagem onde cada célula tem apenas um valor, representando o tom de cinza daquela célula.

Para transformar uma imagem colorida em uma imagem em tons de cinza podemos, entre outras técnicas, para cada célula, tirar a mádia dos valores dos 3 canais.


\textbf{Exercício}: Dada a imagem representada pela tabela abaixo, com cores variando entre $[0, 255]$, onde cada célula tem uma tupla representando os valores de $(r, g, b)$ respectivamente

\begin{tabular}{|c|c|c|}
\hline
$(255, 255,   0)$ & $(  0, 255, 255)$ & $(255,   0, 255)$ \\ \hline
$(122, 122, 122)$ & $(  0,   0,   0)$ & $(255, 255, 255)$ \\ \hline
$(255,   0,   0)$ & $(  0, 255,   0)$ & $(  0,   0, 255)$ \\ \hline

\end{tabular}



\break



\section{Histograma}

Dado um espaço amostral e um conjunto de fenômenos ocorridos, um histograma é um mapeamento entre cada fenômeno possível e a quantidade de vezes que aquele evento ocorre.

Em uma imagem, um histograma é um mapeamento entre cada cor possível e a quantidade de vezes que aquela cor ocorre na imagem.

\textbf{Exercício}: Dada a imagem representada pela tabela abaixo, com cores variando entre $[0, 9]$, onde cada célula tem um valor representando um tom de cinza, faça o histograma desta imagem.

\begin{tabular}{|c|c|c|c|c|}
\hline
$8$ & $1$ & $1$ & $0$ & $4$ \\ \hline
$1$ & $7$ & $1$ & $4$ & $0$ \\ \hline
$1$ & $1$ & $6$ & $1$ & $0$ \\ \hline
$1$ & $7$ & $1$ & $5$ & $0$ \\ \hline
$8$ & $1$ & $1$ & $0$ & $4$ \\ \hline

\end{tabular}





\break





% \section{Segmentação de Imagens}

% Segmentar uma imagem é separar a imagem em regiões significativas. Nesta lista vamos segmentar o espaço de cor de uma imagem em duas cores possíveis. A isto chamamos de segmentação binária. Para isto precisamos escolher um limitante $L$ em que todas as cores maiores que $L$ se tornam a cor de valor máximo, e todas as cores menores se tornam a cor de valor mínimo.

% Uma técnica para fazer uma segmentação binária é escolher um limitante tal que minimize o valor de 
% \[\left| \sum_{i<L}h(i) - \sum_{i>L}h(i)\right|\]

% Onde $h$ é o histograma da imagem

% \textbf{Exercício}: Ainda sobre a imagem anterior. Faça uma segmentação binária na imagem. Use o método de escolha de limitante que minimize o histograma acumulado.

% \begin{tabular}{|c|c|c|c|c|}
% \hline
% $8$ & $1$ & $1$ & $0$ & $4$ \\ \hline
% $1$ & $7$ & $1$ & $4$ & $0$ \\ \hline
% $1$ & $1$ & $6$ & $1$ & $0$ \\ \hline
% $1$ & $7$ & $1$ & $5$ & $0$ \\ \hline
% $8$ & $1$ & $1$ & $0$ & $4$ \\ \hline

% \end{tabular}







% \break





\section{Convoluções (ou Máscaras ou Filtros)}

Uma máscara é uma matrix quadrada de dimenção ímpar que pode ser aplicada em uma imagem

Aplicar uma máscara em uma imagem consiste em gerar uma imagem nova, em que cada célula da imagem nova é criada à partir da célula de mesmo índice na imagem antiga.

Para aplicar uma máscara em uma célula é preciso colocar o centro da máscara na célula e multiplicar cada valor da máscara pelo valor da célula da imagem que este valor coincidir, depois somar todos os valores, limitando os valores pelos valores de máximo e mínimo das cores da imagem.

\textbf{Exercício}: Considere a seguinte imagem com cores entre $[0, 9]$.

\begin{tabular}{|c|c|c|c|c|c|}
\hline
$0$ & $0$ & $0$ & $9$ & $9$ & $9$ \\ \hline
$0$ & $0$ & $0$ & $9$ & $9$ & $9$ \\ \hline
$0$ & $2$ & $0$ & $9$ & $9$ & $9$ \\ \hline
$0$ & $0$ & $0$ & $9$ & $9$ & $9$ \\ \hline
$0$ & $0$ & $0$ & $9$ & $8$ & $9$ \\ \hline
$0$ & $0$ & $0$ & $9$ & $9$ & $9$ \\ \hline

\end{tabular}

Passe a seguinte máscara na imagem acima

\[
\frac{1}{9}
\begin{bmatrix}
1 & 1 & 1\\
1 & 1 & 1\\
1 & 1 & 1
\end{bmatrix}
\]

\break

\textbf{Exercício}: Considere a seguinte imagem com cores entre $[0, 9]$.

\begin{tabular}{|c|c|c|c|c|c|}
\hline
$0$ & $0$ & $0$ & $9$ & $9$ & $9$ \\ \hline
$0$ & $0$ & $0$ & $9$ & $9$ & $9$ \\ \hline
$0$ & $0$ & $0$ & $9$ & $9$ & $9$ \\ \hline
$0$ & $0$ & $0$ & $9$ & $9$ & $9$ \\ \hline
$0$ & $0$ & $0$ & $9$ & $9$ & $9$ \\ \hline
$0$ & $0$ & $0$ & $9$ & $9$ & $9$ \\ \hline

\end{tabular}

Passe cada uma das seguintes máscaras na imagem acima, separadamente.

\begin{enumerate}[a)]

\item

\[
\begin{bmatrix}
 1 &  2 &  1\\
 0 &  0 &  0\\
-1 & -2 & -1
\end{bmatrix}
\]

\item

\[
\begin{bmatrix}
 1 &  0 & -1\\
 2 &  0 & -2\\
 1 &  0 & -1
\end{bmatrix}
\]

\end{enumerate}



\end{document}
