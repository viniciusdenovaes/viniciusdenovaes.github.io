\documentclass[12pt]{article}

\usepackage[brazil]{babel}   
\usepackage[utf8]{inputenc}  
%\usepackage[T1]{fontenc}
\usepackage{amsfonts,amsmath,amssymb,latexsym} 
\usepackage{float}
\usepackage{graphicx}
\usepackage{enumerate}
\graphicspath{ {images/} }

\def\cQ{\mathcal{Q}}
\def\ve{\varepsilon}

\def\emptyset{\varnothing}



\title{Lista 04 de Linguagens Formais e Autômatos}
\date{}
%\date{2º Período de 2023}
\author{Turma do 3º ano}

\begin{document} 

\maketitle

\begin{description}

\item[Definição de uma Gramática Livre de Contexto (GLC)]: Uma gramática livre de contexto, definida pela 4-tupla
\[G = (\Lambda, \Sigma, \Delta, I)\]
\begin{itemize}
\item Um conjunto finito de variáries $\Lambda$, cada variável repreenta uma linguagem.
\item Um conjunto finito de símbolos terminais $\Sigma$, que forma as strings da linguagem.
\item Um conjunto finito de regras $\Delta$. Cada regra consiste em:
  \begin{itemize}
  \item Uma variável, chamada de cabeça da regra
  \item Um símbolo de produção da regra $\rightarrow$
  \item Uma string de zero ou mais terminais e variáveis, chamada de corpo da regra
  \end{itemize}
  Em cada passo, substituímos uma das variáveis da string que esteja numa das cabeças de uma regra, e subtituímos esta variável pela string que está no corpo desta regra.
\item Uma das variáveis representa a linguagem que está sendo definidia, ela é chamada de símbolo de início, $I\in \Lambda$
\end{itemize}

\item[Derivação]: Uma derivação é a aplicação das regras em uma variável inicial. Podemos representar uma derivação com o símbolo $\Rightarrow$, colocando à esquerda a string original e à direita a string resultante.

\end{description}

\vspace{3em}



\begin{enumerate}



\item Projete gramáticas livre de contexto para as seguintes linguagens:

\begin{enumerate}

\item $\ve$

\item $\{a,b\}^*$

\item O conjunto $\{0^n1^n~|~n\geq 1\}$ (o conjunto de todas as strings de um ou mais 0, seguidos por uma quantidade igual de 1)

\item O conjunto $\{waw^R~|~w\in \{a,b\}^*\}$ (o conjunto de todas as strings que tem um $a$ no meio, e o sufixo depois deste $a$ invertido é igual ao prefixo antes do $a$)

\item O conjunto $\{ww^R~|~w\in \{a,b\}^*\}$ (o conjunto de todas as strings que podem ser escritas como uma string $w$ concatenada com esta string $w$ invertida)

\item O conjunto $\{w~|~w\in \{a,b\}^*, w = w^R\}$ (o conjunto de todas as strings que são iguais a elas mesma invertida)

\item O conjunto $\{0^m1^n~|~m\geq n\}$ (o conjunto de 0 seguidos de 1 tal que a quantidade de 0 seja maior que a quantidade de 1)

\item O conjunto $\{0^m1^n~|~m\leq 2n\}$ 

\item O conjunto $\{uawb~|~u,w\in \{a,b\}^*, |u| = |w|\}$ 

\item $\{w~|~$ $w\in \{0,1,+,(,),^*\}$ e $w$ é uma expressão regular em $\{0,1\}$ $\}$

\item O conjunto de todas as strings em $\{0,1\}^*$ com a mesma quantidade de 0's e 1's.


\end{enumerate}


\item Considere a gramática livre de contexto $G = (\Lambda, \Sigma, \Delta, I)$, onde 
\begin{description}
  \item $\Lambda = \{A,B,S\}$
  \item $\Sigma = \{a,b\}$
  \item $\Delta:$
    \begin{enumerate}[(1)]
    \item $S\rightarrow aB$
    \item $S\rightarrow bA$
    \item $A\rightarrow a$
    \item $A\rightarrow aS$
    \item $A\rightarrow BAA$
    \item $B\rightarrow b$
    \item $B\rightarrow bS$
    \item $B\rightarrow ABB$
    \end{enumerate}
\end{description}
Mostre que as seguintes strings fazem parte desta linguagem

\begin{enumerate}

\item $ababba$

\item $aaabbb$

\item $aaabbbab$

\item $bababaab$

\end{enumerate}



\item Considere a gramática livre de contexto $G = (\Lambda, \Sigma, \Delta, I)$, onde 
\begin{description}
  \item $\Lambda = \{S,A\}$
  \item $\Sigma = \{a,b\}$
  \item $\Delta:$
    \begin{enumerate}[(1)]
    \item $S\rightarrow AA$
    \item $A\rightarrow AAA$
    \item $A\rightarrow a$
    \item $A\rightarrow bA$
    \item $A\rightarrow Ab$
    \end{enumerate}
\end{description}
Mostre que as seguintes strings fazem parte desta linguagem

\begin{enumerate}

\item $babbab$

\item $bbabbabb$

\item $bbbabab$

\end{enumerate}

Forneça um algoritmo para gerar a string $b^mab^nab^p$, para qualquer $m$, $n$ e $p$.




\end{enumerate}

\end{document}

