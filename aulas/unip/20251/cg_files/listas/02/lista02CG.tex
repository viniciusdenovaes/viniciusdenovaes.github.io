\documentclass[12pt]{article}

\usepackage[utf8]{inputenc}  
\usepackage[portuguese]{babel}   


\usepackage{enumerate}

\newcommand{\linesnumbered}{\LinesNumbered}
\newcommand{\dontprintsemicolon}{\DontPrintSemicolon}
\newcommand{\incmargin}{\IncMargin}
\newcommand{\decmargin}{\DecMargin}

\def\le{\leqslant}\def\leq{\le}
\def\ge{\geqslant}\def\geq{\ge}
\newcommand{\floor}[1]{\left\lfloor{#1}\right\rfloor}
\newcommand{\ceil}[1]{\left\lceil{#1}\right\rceil}
\newcommand{\pare}[1]{\left({#1}\right)}
\newcommand{\set}[1]{\left\{{#1}\right\}}
\newcommand{\range}[2]{\set{{#1},\dots,{#2}}}
\newcommand{\dist}{\mathrm{dist}}
\newcommand{\cluster}{\mathrm{cluster}}
\newcommand{\merge}{\mathrm{merge}}



\title{Lista 02 de Computação Gráfica}
\date{1º Período de 2023}
\author{Turma do 4º ano}

\begin{document}

\maketitle

\begin{enumerate}


\item Calcule a matriz das seguintes transformações nas coordenadas homogêneas:

\begin{enumerate}

\item A matriz responsável por fazer uma rotação de 90º.

\item A matriz responsável por fazer uma escala de (1,2).

\item A matriz responsável por fazer uma rotação de 90º em torno do ponto (1,2).

\item A matriz responsável por fazer uma escala de (2,1) em relação ao ponto (1,2).

\end{enumerate}

\item Calcule a fórmula geral (a matriz) das seguintes transformações:

\begin{enumerate}

\item A matriz responsável por fazer uma rotação de $\theta$ em torno do ponto ($x$,$y$).

\item A matriz responsável por fazer uma escala de ($S_x$, $S_y$) em relação ao ponto ($x$,$y$).

\end{enumerate}


\end{enumerate}

\end{document}

