\documentclass[12pt]{article}

\usepackage[utf8]{inputenc}  
\usepackage[portuguese]{babel}   

\usepackage{amsmath}
\usepackage{enumerate}

\newcommand{\linesnumbered}{\LinesNumbered}
\newcommand{\dontprintsemicolon}{\DontPrintSemicolon}
\newcommand{\incmargin}{\IncMargin}
\newcommand{\decmargin}{\DecMargin}

\def\le{\leqslant}\def\leq{\le}
\def\ge{\geqslant}\def\geq{\ge}
\newcommand{\floor}[1]{\left\lfloor{#1}\right\rfloor}
\newcommand{\ceil}[1]{\left\lceil{#1}\right\rceil}
\newcommand{\pare}[1]{\left({#1}\right)}
\newcommand{\set}[1]{\left\{{#1}\right\}}
\newcommand{\range}[2]{\set{{#1},\dots,{#2}}}
\newcommand{\dist}{\mathrm{dist}}
\newcommand{\cluster}{\mathrm{cluster}}
\newcommand{\merge}{\mathrm{merge}}



\title{Lista 03 de Computação Gráfica}
\date{}
\author{Turma do 3º ano}

\begin{document}

\maketitle

\begin{enumerate}


\item Considere a equação paramétrica do círculo de raio 2 e centro em (2,4).
Faça uma translação do centro para a origem, uma escala de 2 no eixo $x$ e uma translação novamente do centro para o seu ponto original.
Qual é a equação paramétrica da curva resultante?


\item Enumere 4 pontos que estão no círculo dado pela equação da curva paramétrica

\[c:
\begin{cases}
	x=2+2\cos(t)\\
	y=4+2\sin(t)
\end{cases}
t\in \left[0, 2\pi\right]
\]


\item Enumere 4 pontos que estão dentro da área definida pela equação $c: (x-2)^2 + (y-4)^2 = 2^2$


\item Enumere 5 pontos que estão na reta que liga e termina nos pontos $(2, 2)$ e $(6, 4)$.


\item Escreva a equação da reta que liga e termina nos pontos $(2, 2)$ e $(6, 4)$






\end{enumerate}

\end{document}

